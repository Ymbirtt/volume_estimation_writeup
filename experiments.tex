\section{Mixing time}\label{sec_mix}

Our first experiment will attempt to bound the mixing time of the Hit and Run random walk in a convex shape. The hit and run walk is known to mix rapidly %cite hit and run is fast and fun, as well as Vempala's proof of warm start-ness
in the sense that the number of steps needed for the walk to reach a uniform ditribution is $O^{*}(n^2)$, however this result actually shows that the walk can only be guaranteed to mix after over $10^{10} n^2$
steps. A hit and run step requires the generation of $n$ uniformly distributed random numbers, which is already a difficult task. Performing $10^{10} n^2$ steps, then, is very much infeasible on modern hardware.

We shall judge the walk to be well-mixed if 1,000 independent points generated by the walk with a fixed starting point at the origin pass a $\chi$-squared test at the 5\% level against the null hypothesis that the points are identically distributed to a set of 1,000 known uniformly distributed points over the shape. It is not clear which shape is likely to constitute a pathological case for the hit and run walk; the ice cream has the smallest volume, and so in a sense fewer points need to be chosen from, but also has the tighest possible corner, in which the walk is most likely to get stuck. By contrast, the ball has the largest volume and no corners. We will test both shapes. Uniformly distributed random points can be generated from a ball fairly easily, and uniform random points can be generated from an ice cream by simple rejection sampling.

\section{Threads per stage}\label{sec_error}

We will now see how the number of threads per phase influences the accuracy of an estimate. In the paper describing our particular estimation method, it is stated that, for an n-dimensional shape, in order to achieve an error of $\varepsilon$, we should use $400\frac{n\log n}{\varepsilon^2}$ threads per stage. . We shall vary the number of threads used per stage, and see how the distribution of estimates varies with this value. As before, we will use both the ice cream and the ball to respectively maximise and minimise the ratios between successive shapes. Our analysis here will be largely visual.