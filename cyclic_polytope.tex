Using an LP solver, an oracle query takes time polynomial in the number of verticies in the convex shape. The shape is represented by the vertices on its convex hull, which is referred to as v-representation. The major alternative is the h-representation, in which the shape is represented as a series of halfspaces - one for each facet. It has been proven \cite{Buchta85} that the convex hull of $v$ uniformly random vertices from an $n$-ball has, on average, $O(v)$ facets, so either representation is roughly equivalent for a polytope selected in this way. The cross-polytope which we are using as a base has $2n$ vertices and $2^n$ facets, so a vertex representation can be considered ``fast" for these purposes. Can we construct a pathological case where there are exponentially many vertices and facets? The probability of hitting a pathological case is sufficiently small that it can justifiably be ignored when considering random inputs, however it is important discuss what would happen in one of these pathological cases.

The moment curve in $\arr^n$ is defined:

$$
\tengm(t) = (t, t^2, t^3, ..., t^n)
$$

A Cyclic Polytope is the convex hull of some sequence of points on $\tengm$. By McMullen's Upper Bound Theorem \cite{McMullen70}, a cyclic polytope with $f$ facets has $v$ vertices, with

$$
v = {{f-\lceil n/2 \rceil}\choose{\lfloor {n/2} \rfloor}} + {{f-\lfloor n/2 \rfloor -1} \choose {\lceil {n/2} \rceil - 1}}
$$

Alternatively, a cyclic polytope with $f$ facets has $\Theta(f^{n/2})$ vertices. From this, we can construct a convex hull with an exponential number of both vertices and facets, for which neither representation can be of polynomially bounded size in $n$. From my research, no polynomial-time method could be found for testing membership of cyclic polytopes. On this basis, I would call the validity of the assumption of a polynomial time oracle into question. 