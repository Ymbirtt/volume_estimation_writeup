\begin{titlepage}

\centering{\textsc{\huge{On the Properties of Monte Carlo Estimation of the Volume of a Convex Polytope}}}\\[1.5cm]

\centering{\textsc{\large{A Summer Project Funded by the University of Bristol's Prize Studentship Scheme}}}\\[1.5cm]

\begin{minipage}{0.4\textwidth}
\begin{flushleft} \large
\emph{Author:}\\
Tim Jones
\end{flushleft}
\end{minipage}
\begin{minipage}{0.4\textwidth}
\begin{flushright} \large
\emph{Supervisor:} \\
Dr. Rapha\"{e}l Clifford
\end{flushright}
\end{minipage}\\[1.5cm]

\vfill

\begin{abstract}
We evaluate a practical implementation of an efficient convex volume estimation algorithm, and tune its parameters such that they produce accurate results. We find that $2^5$ steps of a hit and run random walk is sufficient to produce adequately random points for shapes up to 10 dimensions and $2^{13}$ random points is enough to provide reasonable quality estimates of volume ratios. We then conclude by showing that for a convex polytope in 10 or more dimensions with 25 or more vertices, it is more efficient to use a randomised algorithm than to compute the volume exactly.
\end{abstract}

\vfill

Rendered on \today

\end{titlepage}

\clearpage